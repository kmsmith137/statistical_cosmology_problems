\documentclass[aps,prd,superscriptaddress,groupedaddress,nofootinbib,nobibnotes]{revtex4}

\usepackage{graphicx}
\usepackage{dcolumn}
\usepackage{bm}
\usepackage{amssymb}
\usepackage{epstopdf}
\usepackage{amsmath}
\usepackage{amsfonts}
\usepackage{color}
\usepackage{mathrsfs}
% \usepackage{comment}
% \usepackage{url}
% \usepackage{wick}
% \usepackage{feynmp}
% \usepackage{braket}

\setlength{\parindent}{20pt}
% \setlength{\parskip}{1mm}

\setcounter{topnumber}{1}    % default value is 2.
\setcounter{bottomnumber}{0} % default value is 1.

\hyphenation{ALPGEN}
\hyphenation{EVTGEN}
\hyphenation{PYTHIA}

\newcommand{\kms}[1]{\textcolor{blue}{(KMS: #1)}}
\newcommand{\be}{\begin{equation}}
\newcommand{\ee}{\end{equation}}
\newcommand{\ba}{\begin{eqnarray}}
\newcommand{\ea}{\end{eqnarray}}
\newcommand{\nn}{\nonumber}
\newcommand{\barr}{\begin{array}}
\newcommand{\earr}{\end{array}}
\newcommand{\eqdef}{\stackrel{\rm def}{=}}
\newcommand{\bigoh}{\mathcal{O}}

\newcommand\lsim{\mathrel{\rlap{\lower4pt\hbox{\hskip1pt$\sim$}}
        \raise1pt\hbox{$<$}}}
\newcommand\gsim{\mathrel{\rlap{\lower4pt\hbox{\hskip1pt$\sim$}}
        \raise1pt\hbox{$>$}}}

\def\threej#1#2#3#4#5#6{\left( \begin{array}{ccc} #1 & #2 & #3 \\ #4 & #5 & #6 \end{array} \right) }
\def\smallsum{\mathop{\textstyle\sum}\limits}
\def\Var{\mbox{Var}}
\def\Cov{\mbox{Cov}}
\def\x{{\bf x}}
\def\l{{\bf l}}
\def\L{{\bf L}}
\def\hphi{{\hat\phi}}

\renewcommand{\baselinestretch}{1.1}

\begin{document}

\title{CMB lensing problems}

\author{Kendrick~M.~Smith}
\affiliation{Perimeter Institute for Theoretical Physics, Waterloo, ON N2L 2Y5, Canada}

\date{\today}

% \begin{abstract}
% ABSTRACT HERE
% \end{abstract}
% \pacs{}

\maketitle

\par\noindent
In this version of the problems, we make the following simplifying assumptions:
\begin{enumerate}
\item We assume an infnite-area flat sky.
\item We only consider CMB temperature (not polarization).
\item The CMB lensing deflection field ${\bf d}(\x)$ is assumed to be curl-free,
  i.e.~${\bf d} = \nabla \phi$ for some choice of potential
  $\phi(\x)$.  (This is true to first order in the cosmological perturbations,
  but not higher orders.)
\item We assume that the {\em unlensed} CMB $T_{\rm unl}$ and lensing potential $\phi$
  are statistically independent.  (This is an excellent approximation on most angular scales,
  but breaks down on small scales and very large scales, due to the ISW effect.)
\end{enumerate}

\section{Definitions and notation}

Let $T_{\rm unl}(\x)$ be the unlensed CMB temperature.
Our Fourier conventions are:
\be
T_{\rm unl}(\x) = \int \frac{d^2\l}{(2\pi)^2} \, T_{\rm unl}(\l) e^{i\l\cdot\x}
\hspace{1.5cm}
T_{\rm unl}(\l) = \int d^2\x \, T_{\rm unl}(\x) e^{-i\l\cdot\x}  \label{eq:continuous_fourier_conventions}
\ee
The unlensed CMB is a Gaussian field, i.e. its two-point function is:
\be
\big\langle T_{\rm unl}(\l) T_{\rm unl}(\l')^* \big\rangle = C_l^{\rm unl} \, (2\pi)^2 \delta^2(\l-\l')  \label{eq:continuous_cl}
\ee
and higher-point correlation functions are obtained by applying Wick's theorem.
The unlensed CMB power spectrum $C_l^{\rm unl}$ can be computed numerically using
CAMB or CLASS.

The lensed CMB is given by:
\be
T_{\rm len}(\x) = T_{\rm unl}\big( \x + \nabla\phi(\x) \big)  \label{eq:tlen_def}
\ee
where the field $\phi(\x)$ is the CMB lensing potential.
That is, the lensed CMB is obtained from the unlensed CMB by remapping points
on the sky, via a deflection field $\x \rightarrow \x + {\bf d}(\x)$ which is
curl-free (i.e.~${\bf d} = \nabla\phi$ for some choice of potential $\phi$).
The power spectrum $C_l^{\phi\phi}$ can be computed numerically
using CAMB/CLASS.  However, $\phi$ is not a Gaussian field,
so the power spectrum $C_l^{\phi\phi}$ does not fully
characterize it.

Since CMB lensing deflections are small (a typical deflection
angle $|\nabla\phi|$ is a few arcminutes), we often expand lensing-related
quantities in powers of $\phi$.  For example, the lensed CMB map in Eq.~(\ref{eq:tlen_def}) can be
expanded as follows:
\be
T_{\rm len}(\x) = T_{\rm unl}(\x) + \big( \partial_i\phi(\x) \big) \big( \partial_i T_{\rm unl}(\x) \big)
 + \frac{1}{2} \big( \partial_i\phi(\x) \big) \big( \partial_j(\phi(x) \big) \big(\partial_i \partial_j T_{\rm unl}(\x)\big)
 + \bigoh(\phi^3)
\ee
The power spectrum $C_l^{\rm len}$ of the lensed CMB is not equal
to $C_l^{\rm unl}$!  The lensed power spectrum $C_l^{\rm len}$ can also be computed
using CAMB/CLASS.  (In problem 2 below, you'll calculate $C_l^{\rm len}$
to order $\bigoh(\phi^2)$.)

Let $T_{\rm obs}$ be the observed CMB:
\be
T_{\rm obs}(\x) = T_{\rm len}(\x) + N(\x)
\ee
where $N(\x)$ is the instrumental noise.  We will assume that $N(\x)$ is an isotropic
Gaussian field with power spectrum $N_l$ (the noise power spectrum).  This is a common
assumption in theory/forecasting papers, but real experiments have complicated noise models!
We will denote the power spectrum of the observed CMB by $C_l^{\rm tot} = C_l^{\rm len} + N_l$.

\section{Theory problems}
\label{sec:theory_problems}

\par\noindent
Problems in this section are pencil-and-paper calculations.  The next section (\S\ref{sec:computer_problems}) contains computer problems.

\begin{enumerate}

\item {\bf Two-point function of T in a fixed lens.}
  Let $\langle \cdot \rangle_T$ denote an expectation value over random realizations
  of the {\em unlensed} CMB temperature, in a fixed realization of the lensing potential $\phi$.
  Show that:
\ba
\big\langle T_{\rm len}(\l_1) T_{\rm len}(\l_2) \big\rangle_T
&=& C_{l_1}^{\rm unl} (2\pi)^2 \delta^2(\l_1+\l_2) + K_1(\l_1,\l_2) \phi(\l_1+\l_2) \nn \\
&& \hspace{0.5cm} + \int \frac{d^2\l}{(2\pi)^2} \, K_2(\l,\l_1,\l_2) \phi(\l_1-\l) \phi(\l_2+\l) + \bigoh(\phi^3) \label{eq:2pf_fixed_lens}
\ea
where the first-order and second-order kernels $K_1,K_2$ are defined by:
\ba
K_1(\l_1,\l_2) &=& -\big[ \l_1\cdot(\l_1+\l_2) \big] C_{l_1}^{\rm unl} - \big[ \l_2\cdot(\l_1+\l_2) \big] C_{l_2}^{\rm unl} \\
K_2(\l,\l_1,\l_2) &=&
\frac{1}{2} \big[ \l_2 \cdot(\l_1-\l) \big] \big[ \l_2 \cdot (\l_2+\l) \big] C_{l_2}^{\rm unl}
+ \frac{1}{2} \big[ \l_1 \cdot(\l_1-\l) \big] \big[ \l_1 \cdot (\l_2+\l) \big] C_{l_1}^{\rm unl} \nn \\
&& \hspace{0.5cm}
- \big[ \l\cdot(\l_1-\l) \big] \big[ \l \cdot (\l_2+\l) \big] C_l^{\rm unl}
\ea
Note that most terms on the RHS of Eq.~(\ref{eq:2pf_fixed_lens}) do not contain $\delta^2(\l_1+\l_2)$.
This is because translation invariance is broken by the fixed lens realization.

\item {\bf Lensed CMB power spectrum.}
By averaging the RHS of Eq.~(\ref{eq:2pf_fixed_lens}) over $\phi$, show that
\be
\big\langle T_{\rm len}(\l_1) T_{\rm len}(\l_2) \big\rangle = C_l^{\rm len} \, (2\pi)^2 \delta^2(\l_1+\l_2)  \label{eq:2pf_lens_averaged}
\ee
where the lensed CMB power spectrum $C_l^{\rm len}$ is:
\be
C_l^{\rm len} = C_l^{\rm unl} + \int \frac{d^2\l'}{(2\pi)^2} \, \bigg(
\big[ \l' \cdot (\l-\l') \big]^2 C_{l'}^{\rm unl} - \big[ \l \cdot (\l-\l') \big]^2 C_l^{\rm unl}
\bigg) C_{|\l-\l'|}^{\phi\phi} + \bigoh(\phi^3)
\ee
In Eq.~(\ref{eq:2pf_lens_averaged}), $\langle \cdot \rangle$ denotes an expectation value
of realizations of both the unlensed CMB and the lensing potential.
Note that the delta function $\delta^2(\l_1+\l_2)$ now appears on the RHS of Eq.~(\ref{eq:2pf_lens_averaged}),
since translation invariance is restored by averaging over the lens realization.

\item {\bf Quadratic estimator.}
In this megaproblem, we'll derive the quadratic estimator $\hphi$ for reconstructing the lensing potential from the CMB.

The idea is as follows.  Consider the first two terms in Eq.~(\ref{eq:2pf_fixed_lens}):
\be
\big\langle T_{\rm len}(\l_1) T_{\rm len}(\l_2) \big\rangle_T = C_{l_1}^{\rm unl} (2\pi)^2 \delta^2(\l_1+\l_2) + K_1(\l_1,\l_2) \phi(\l_1+\l_2) + \cdots
\ee
If $\l_1+\l_2 \ne 0$, we can think of the product $(T_{\rm obs}(\l_1) T_{\rm obs}(\l_2))$ as a noisy estimator
of $\phi(\l_1+\l_2)$, since its expectation value is proportional to $\phi(\l_1+\l_2)$.  For a single pair of
Fourier modes $(\l_1,\l_2)$, this would be a very noisy estimate of $\phi(\l_1+\l_2)$.  However, for a fixed
lensing mode $\phi(\L)$, we can sum over all pairs $(\l_1,\l_2)$ satisfying $\l_1+\l_2=\L$, to get an estimate
of $\phi(\L)$ with interesting SNR.

More formally, we consider an estimator of the form:
\be
\hphi(\L) = \int_{\l_1+\l_2=\L} W(\l_1,\l_2) \, T_{\rm obs}(\l_1) T_{\rm obs}(\l_2)
\ee
with weight function $W(\l_1,\l_2)$ to be determined shortly.
Here, we have used the notation:
\be
\int_{\l_1+\l_2 = \L} \equiv \int \frac{d^2\l_1}{(2\pi)^2} \, \frac{d^2\l_2}{(2\pi)^2} \, (2\pi)^2 \delta^2(\l_1+\l_2-\L)
\ee
Without loss of generality, we assume $W(\l_1,\l_2) = W(\l_2,\l_1)$ and $W(\l_1,\l_2)^* = W(-\l_1,-\l_2)$.  (Why is this justified?)

Show that to first order in $\phi$, the expectation value $\langle \hphi(\L) \rangle_T$ is given by:
\be
\langle \hphi(\L) \rangle_T = \int_{\l_1+\l_2=\L} W(\l_1,\l_2) \big[ -\l_2\cdot(\l_1+\l_2) \big] C_{l_2}^{\rm unl} + \bigoh(\phi^2)
\ee
Also show that to zeroth order in $\phi$, the power spectrum of $\hphi$ is given by:
\be
\langle \hphi(\L) \hphi(\L')^* \rangle = N_L^{(0)} \, (2\pi)^2 \delta^2(\L-\L') + \bigoh(\phi^2)
\ee
where we have defined the $N^{(0)}$-bias by:
\be
N_L^{(0)} \equiv \frac{1}{2} \int_{\l_1+\l_2=\L} |W(\l_1,\l_2)|^2 C_{l_1}^{\rm tot} C_{l_2}^{\rm tot}  \label{eq:n0_w}
\ee
(The $N^{(0)}$ bias is just the power spectrum of $\hphi$ if no lensing signal is present.
The motivation for the name ``$N^{(0)}$-bias'' will be explained later!)

Now we ask: what choice of weight function $W(\l_1,\l_2)$ minimzes the $N^{(0)}$ bias in Eq.~(\ref{eq:n0_w}),
subject to the constraint that $\langle \hphi(\L) \rangle_T = \phi(\L)$, i.e. $\hphi$ is an unbiased estimator of $\phi$?
By solving a constrained optimization problem for $W$, arrive at the following explicit formulas
for $\hphi(\L)$ and $N_L^{(0)}$:
\ba
N_L^{(0)} &=& \int_{\l_1+\l_2=\L} \frac{
  \big[ ( \l_1\cdot \L ) C_{l_1}^{\rm unl} \big]
  \big[ ( \l_1\cdot \L ) C_{l_1}^{\rm unl} + ( \l_2\cdot \L ) C_{l_2}^{\rm unl} \big]}{C_{l_1}^{\rm tot} C_{l_2}^{\rm tot}}  \label{eq:n0_fourier} \\
\hphi(\L) &=& N_L^{(0)} \int_{\l_1+\l_2=\L} \frac{(-\l_1\cdot \L) C_{l_1}^{\rm unl}}{C_{l_1}^{\rm tot} C_{l_2}^{\rm tot}}
  \, T_{\rm obs}(\l_1) T_{\rm obs}(\l_2)  \label{eq:hphi_fourier}
\ea

\item {\bf Position-space representation.}
The Fourier-space representations of $N_L^{(0)}$ and $\hphi(\L)$ in Eqs.~(\ref{eq:n0_fourier}),~(\ref{eq:hphi_fourier})
appear to be computationally expensive.  Imagine that we want to use Eq.~(\ref{eq:hphi_fourier}) to compute every Fourier
mode $\hphi(\L)$, on a finite pixelized flat sky represented as an $(N,N)$-array.  Thus the number of pixels
is $N^2$, and the number of Fourier modes is also $N^2$.  We outer-loop over $\L$ values, and for each $\L$
we inner-loop over pairs $(\l_1,\l_2)$ such that $\l_1+\l_2=\L$.  Both loops have $N^2$ iterations, so the
computational cost appears to be $\bigoh(N^4)$.

It turns out that $N_L^{(0)}$ and $\hphi(\L)$ have position-space representations with computational cost
$\bigoh(N^2 \log N)$, a huge improvement!  First, show that:
\be
N_L^{(0)} = -L_i L_j \int d^2\x \, e^{-i\L\cdot\x} \Big[
  \big( \partial_i \gamma_1(\x) \big) \big( \partial_j \gamma_1(\x) \big) +
  \gamma_0(\x) \big( \partial_i \partial_j \gamma_2(\x) \big) \Big]  \label{eq:n0_position}
\ee
where we have defined the field:
\be
\gamma_p(\l) = \frac{(C_l^{\rm unl})^p}{C_l^{\rm tot}}
\ee
Second, show that:
\be
\hphi(\L) =  N_L^{(0)} (iL_i) \int d^2\x \, e^{-i\l\cdot\x} \Big( \alpha(\x) \, \partial_i \beta(\x) \Big)  \label{eq:hphi_position}
\ee
where we have defined fields:
\be
\alpha(\l) = \frac{1}{C_l^{\rm tot}} T_{\rm obs}(\l)
\hspace{1.5cm}
\beta(\l) = \frac{C_l^{\rm unl}}{C_l^{\rm tot}} T_{\rm obs}(\l)
\ee

\item {\bf $N^{(1)}$-bias.}  Forthcoming!
  
\item {\bf Four-point function of the lensed CMB.}  Forthcoming!

\item {\bf Optimal four-point estimator.}  Forthcoming!

\newcounter{enumi_save}
\setcounter{enumi_save}{\value{enumi}}
\end{enumerate}

\section{Computer problems}
\label{sec:computer_problems}

In the pencil-and-paper calculations in~\S\ref{sec:theory_problems}, an infinite-area, infinite-resolution flat sky was assumed.
On the computer, you'll need to make both the area and resolution finite of course.
I strongly recommend using periodic boundary conditions!
Then the sky is represented by a shape $(N,N$) array, with sky area $\Theta_{\rm sky}^2$ and pixel area $\Theta_{\rm pix}^2 = (\Theta_{\rm sky}/N)^2$.
Allowed Fourier modes are of the form $(l_x, l_y) = (2\pi/\Theta_{\rm sky}) (n_x, n_y)$, where $(n_x,n_y)$ are integers.

When applying equations in~\S\ref{sec:theory_problems}, you'll need to figure out how to ``discretize'' each equation
so that it applies to a finite pixelized sky.  This discretization process is actually fairly mechanical if you choose
a uniform, consistent set of conventions.  In the next few paragraphs, I'll outline the conventions that I use.

First, real-space and Fourier-space integrals are discretized as:
\ba
\int d^2\x \, (\cdots) & \rightarrow & \Theta_{\rm pix}^2 \sum_{\x} (\cdots)  \label{eq:discretize_x_integral}  \\
\int \frac{d^2\l}{(2\pi)^2}  \, (\cdots) & \rightarrow & \frac{1}{\Theta_{\rm sky}^2} \sum_{\l} (\cdots)  \label{eq:discretize_l_integral}
\ea
These rules make intuitive sense: in Eq.~(\ref{eq:discretize_x_integral}), the pixel area $\Theta_{\rm pix}^2$ appears on
the RHS, since the integral on the LHS contains the area element $d^2\x$.
In Eq.~(\ref{eq:discretize_l_integral}), the factor $\Theta_{\rm sky}^{-2}$ appears on the RHS for a similar reason.
The sum over $\l$ can be viewed as summing over an $l$-space ``pixelization'' with pixel area $(2\pi/\Theta_{\rm sky})^2$.
Therefore, the factor $d^2\l/(2\pi)^2$ on the LHS becomes $(2\pi/\Theta_{\rm sky})^2 / (2\pi)^2 = \Theta_{\rm sky}^{-2}$
on the RHS.

Second, real-space and Fourier-space delta functions are discretized as:
\be
\delta^2(\x-\x') \rightarrow \frac{1}{\Theta_{\rm pix}^2} \delta_{\x\x'}
\hspace{1.5cm}
(2\pi)^2 \delta^2(\l-\l') \rightarrow \Theta_{\rm sky}^2 \delta_{\l\l'}  \label{eq:discretize_delta_functions}
\ee
This is required for consistency with our conventions for discretizing integrals in
Eqs.~(\ref{eq:discretize_x_integral}),~(\ref{eq:discretize_l_integral}), since we want the
identities $\int d^2\x \, \delta^2(\x-\x') = \int d^2\l \, \delta^2(\l-\l') = 1$ to hold.

Additional discretization rules can be obtained via systematic application of
Eqs.~(\ref{eq:discretize_x_integral})--(\ref{eq:discretize_delta_functions}).
For example, the Fourier normalization conventions on a finite pixelized sky are:
\be
T(\x) = \frac{1}{\Theta_{\rm sky}^2} \sum_{\l} T(\l) e^{i\l\cdot\x}
\hspace{1.5cm}
T(\l) = \Theta_{\rm pix}^2 \sum_\x T(\x) e^{-i\l\cdot\x}  \label{eq:discrete_fourier_conventions}
\ee
by taking the continuous-sky Fourier conventions in Eq.~(\ref{eq:continuous_fourier_conventions}),
and discretizing the integrals using Eqs.~(\ref{eq:discretize_x_integral}),~(\ref{eq:discretize_l_integral}).

Similarly, the normalization for the power spectrum on a finite pixelized sky is:
\be
\big\langle T(\l) T(\l')^* \big\rangle = C_l \Theta_{\rm sky}^2 \delta_{\l\l'}  \label{eq:discrete_cl}
\ee
by taking the continuous-sky Eq.~(\ref{eq:continuous_cl}), and discretizing
the delta functions using Eq.~(\ref{eq:discretize_delta_functions}).

\begin{enumerate}
\setcounter{enumi}{\value{enumi_save}}
  
\item {\bf Power spectrum of a Gaussian field.}
  Write code to simulate the unlensed CMB temperature $T_{\rm unl}(\x)$ in real space, using the power spectrum $C_l^{\rm unl}$
  from CAMB/CLASS.  Test your simulation code, by writing an ``analysis'' code which estimates the power spectrum of the real-space
  field $T_{\rm unl}(\x)$, and verifying that the power spectrum of your simulations (averaged over many Monte Carlo simulations)
  agrees with the input $C_l^{\rm unl}$.

  This problem is more subtle than it appears!  When you simulate $T_{\rm unl}(\l)$ in Fourier space, you'll want to
  ensure that the real-valuedness condition $T_{\rm unl}(\l)^* = T_{\rm unl}(-\l)$ is satisfied.  If $N$ is even, there
  are a few high-$l$ ``Nyquist'' modes satisfying $\l = -\l$!  How should those be treated?

\item {\bf Simulating the lensed CMB.}
  Now write code to simulate the lensed CMB temperature $T_{\rm len}(\x)$.
  Assume for simplicity that the lensing potential $\phi$ is a Gaussian field.
  Since the lensed CMB is non-Gaussian, it cannot be simulated directly from the power spectrum $C_l^{\rm len}$.
  To correctly capture the non-Gaussian statistics, you'll need simulate $T_{\rm unl}$ and $\nabla\phi$,
  then apply the deflection operation in Eq.~(\ref{eq:tlen_def}).
  
  Estimate the power spectrum of your lensed CMB simulations, and verify agreement with
  the lensed CMB power spectrum $C_l^{\rm len}$ from CAMB/CLASS.  (They will not agree exactly,
  since the CAMB/CLASS calculation of $C_l^{\rm len}$ does not make the flat-sky approximation,
  but the agreement will be very close.)

  A little discussion on implementation issues which may arise!
  
  Here are two ways to simulate $\nabla\phi$ in real space: (1) simulate $\phi$ in real space,
  and apply a real-space gradient operator defined by finite differencing, (2) simulate $\phi$
  in Fourier space and multiply by $(i\l)$ before taking the Fourier transform.
  For this problem, I think either should work, but I like (2) better since it's a bit more
  numerically stable, and turns out to generalize better to the curved-sky case.

  After you simulate $\nabla\phi(\x)$ and $T_{\rm unl}(\x)$ in real space on a regular grid,
  you'll want to compute $T_{\rm len}(\x)$ on a regular grid, by applying the deflection
  operation in Eq.~(\ref{eq:tlen_def}).  This will require you to evaluate $T_{\rm unl}(\x)$
  at a sky location $\x$ which is not part of the grid.  This can be done by interpolation,
  but interpolation requires higher resolution than you might initially expect.

  To quantify this better, suppose we are interested in simulating the lensed CMB on some
  range of scales $l \le l_{\rm max}$.  Naively, you might guess that the necessary angular
  resolution is the Nyquist sampling rate $\Theta_{\rm pix} = \pi/l_{\rm max}$.  However,
  you'll find that for the interpolation operation on the RHS of Eq.~(\ref{eq:tlen_def}),
  you need angular resolution significantly better than this.  (The exact value will
  depend on the interpolation scheme chosen.)  What's a good way to decide what resolution
  to use?

\item {\bf Quadratic estimator.}
  Implement the quadratic estimator $T_{\rm obs}(\l) \rightarrow \hphi(\L)$, using its
  position-space representation in Eq.~(\ref{eq:hphi_position}).

  Apply the quadratic estimator to an ensemble of {\em unlensed} CMB maps.
  Estimate the power spectrum of $\hphi$, and show that it agrees numerically with the
  $N^{(0)}$-bias, which you can compute using its position-space representation in
  Eq.~(\ref{eq:n0_position}).

  Next, fix a random realization of the lensing potential $\hphi$, and construct an
  ensemble of lensed CMB realizations, by randomizing the unlensed CMB.  Show that
  the expectation value $\langle \hphi \rangle_T$ taken over this Monte Carlo ensemble
  agrees with the true lens realization $\phi$.

\end{enumerate}



%\begin{figure}
%\centerline{\includegraphics[width=14cm]{x.pdf}}
%\caption{xxx}
%\label{fig:xxx}
%\end{figure}

% Note: Syntax for clipping a figure is
% \includegraphics[trim={5cm 0 0 0},clip,...]
%  where ordering is <left> <lower> <right> <upper>

% \section*{Acknowledgements}
%
% Research at Perimeter Institute is supported by the Government of Canada
% through Industry Canada and by the Province of Ontario through the Ministry of Research \& Innovation.
% Some computations were performed on the GPC cluster at the SciNet HPC Consortium.
% SciNet is funded by the Canada Foundation for Innovation under the auspices of Compute Canada,
% the Government of Ontario, and the University of Toronto.
% KMS was supported by an NSERC Discovery Grant and an Ontario Early Researcher Award.

% \bibliographystyle{h-physrev}
% \bibliography{xxx}

% \appendix
% \section{Appendix}

\end{document}
