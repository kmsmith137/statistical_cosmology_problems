\documentclass[aps,prd,superscriptaddress,groupedaddress,nofootinbib,nobibnotes]{revtex4}

\usepackage{graphicx}
\usepackage{dcolumn}
\usepackage{bm}
\usepackage{amssymb}
\usepackage{epstopdf}
\usepackage{amsmath}
\usepackage{amsfonts}
\usepackage{color}
\usepackage{mathrsfs}
% \usepackage{comment}
% \usepackage{url}
% \usepackage{wick}
% \usepackage{feynmp}
% \usepackage{braket}

\setlength{\parindent}{20pt}
% \setlength{\parskip}{1mm}

\setcounter{topnumber}{1}    % default value is 2.
\setcounter{bottomnumber}{0} % default value is 1.

\hyphenation{ALPGEN}
\hyphenation{EVTGEN}
\hyphenation{PYTHIA}

\newcommand{\kms}[1]{\textcolor{blue}{(KMS: #1)}}
\newcommand{\be}{\begin{equation}}
\newcommand{\ee}{\end{equation}}
\newcommand{\ba}{\begin{eqnarray}}
\newcommand{\ea}{\end{eqnarray}}
\newcommand{\nn}{\nonumber}
\newcommand{\barr}{\begin{array}}
\newcommand{\earr}{\end{array}}
\newcommand{\eqdef}{\stackrel{\rm def}{=}}
\newcommand{\bigoh}{\mathcal{O}}

\newcommand\lsim{\mathrel{\rlap{\lower4pt\hbox{\hskip1pt$\sim$}}
        \raise1pt\hbox{$<$}}}
\newcommand\gsim{\mathrel{\rlap{\lower4pt\hbox{\hskip1pt$\sim$}}
        \raise1pt\hbox{$>$}}}

\def\threej#1#2#3#4#5#6{\left( \begin{array}{ccc} #1 & #2 & #3 \\ #4 & #5 & #6 \end{array} \right) }
\def\smallsum{\mathop{\textstyle\sum}\limits}
\def\Var{\mbox{Var}}
\def\Cov{\mbox{Cov}}
\def\x{{\bf x}}
\def\y{{\bf y}}
\def\r{{\bf r}}
\def\k{{\bf k}}

\renewcommand{\baselinestretch}{1.1}

\begin{document}

\title{Statistical cosmology problems}

\author{Kendrick~M.~Smith}
\affiliation{Perimeter Institute for Theoretical Physics, Waterloo, ON N2L 2Y5, Canada}

\date{\today}

% \begin{abstract}
% ABSTRACT HERE
% \end{abstract}
% \pacs{}

\maketitle

\section{Power spectra in flat space}

\par\noindent
In this section, let $\phi(\x)$ be a 3D random field with power spectrum $P(k)$ defined by
\be
\langle \phi_{\k}^* \phi_{\k'} \rangle = P(k) \, (2\pi)^3 \delta^3(\k-\k')  \label{eq:pk_def}
\ee

\begin{enumerate}

\item What symmetry assumptions are made in writing down Eq.~(\ref{eq:pk_def})?
 Show how Eq.~(\ref{eq:pk_def}) follows from these assumptions.

\item Suppose we define a new random field $\psi$ by $\psi(\x) = \nabla^2 \phi(\x)$.
What is the power spectrum $P_\psi(k)$ of the $\psi$ field (in terms of the original power spectrum $P(k)$)?

\item {\bf Scaling.}
Suppose that we define the field $\psi(\x) = \phi(\lambda\x)$ by rescaling coordinates,
where $\lambda$ is a constant.  Show that the power spectrum scales as
\be
P_\psi(k) = \lambda^{-3} P(\lambda^{-1} k)
\ee
Note that as a consequence, if a field's power spectrum is proportional to $k^{-3}$, 
then it is {\em scale-invariant}, in the sense that magnifying/demagnifying does 
not change its statistical properties.

\item {\bf Slicing.}
Suppose we define a {\em two-dimensional} random field $\psi(\x)$ by simply restricting 
the 3D field $\phi$ to a 2D plane, i.e. in real space:
\be
\psi(x,y) = \phi(x,y,0)
\ee
Let $P_\psi(l)$ be the power spectrum of this field.  (Note that we use $l$ for Fourier wavenumbers in 2D, and
$k$ in 3D.)  Show that $P_\psi(l)$ is related to the original 3D power spectrum $P(k)$ by
\be
P_\psi(l) = \int_l^\infty dk \, \frac{k}{2\pi\sqrt{k^2-l^2}} P(k)
\ee
Suppose the 3D power spectrum has the scale-invariant form $P(k) = A k^{-3}$.
What is $P_\psi(l)$ in this case?  Is it also scale invariant?

\item {\bf Slicing, part 2.}
In the previous problem, we worked out the transformation law for the power spectrum,
when a 3D field is restricted on a 2D plane.  What is the analogous transformation law
when restricting to a 1D line?

\item {\bf Convolution.}
Suppose we define a new field $\psi(\x)$ by convolving with a kernel in real space.
\be
\psi(\x) = \int d^3\y \, W(|\x-\y|) \phi(\y)
\ee
where $W(r)$ is a kernel which we assume only depends on $r = |\r|$.  How are the Fourier transforms $\tilde\psi(\k)$
and $\tilde\phi(\k)$ related?  How is the power spectrum $P_\psi(k)$ related to $P(k)$?

\item {\bf Tophat averaging.}
Suppose we define a new field $\psi(\x)$ by ``tophat averaging'' the field $\phi$.
That is, the value of $\psi$ at any point $\x$ is given by averaging $\phi$ over a ball of radius $R$ centered at $\x$.
How is the power spectrum $P_\psi(k)$ related to $P(k)$?  (Hint: use the result of the previous problem.)

\end{enumerate}

\end{document}
